%%%%%%%%%%%%%%%%%%%%%%%%%%%%%%%%%%%%%%%%%
% Latex Nice Article
%
% Author(s):
% Marwan Abdellah (abdelah.marwan@gmail.com)
%
% License:
% MIT License 
%
%%%%%%%%%%%%%%%%%%%%%%%%%%%%%%%%%%%%%%%%%

\documentclass[12pt]{article} 				% Generate the final document 
% \documentclass[draft, 12pt]{article} 	% Testing the document without images 

% Load the structure of the template article.
%%%%%%%%%%%%%%%%%%%%%%%%%%%%%%%%%%%%%%%%%
% Latex Nice Article
% Structure specification file.
%
% Author(s):
% Marwan Abdellah (abdellah.marwan@gmail.com)
%
% License:
% MIT License 
%
%%%%%%%%%%%%%%%%%%%%%%%%%%%%%%%%%%%%%%%%%

%%%%%%%%%%%%%%%%%%%%%%%%%%%%%%%%%%%%%%%%%
% Packages 
%%%%%%%%%%%%%%%%%%%%%%%%%%%%%%%%%%%%%%%%%

% Graphics 
\usepackage{graphicx} 
% Appendix 
\usepackage[titletoc,toc]{appendix} 

% Orange notes 
\usepackage{todonotes} 

% Hidden comments
\usepackage{comment} 

% Acronyms 
\usepackage[printonlyused]{acronym} 

% Multicolumn
\usepackage{multicol} 

% Collapse
\usepackage{etoolbox} 
\usepackage[outercaption]{sidecap} 
\usepackage{booktabs}

% References 
\usepackage{xcolor}

% References and links 
\usepackage{hyperref}

% Figure-related packages 
\usepackage{subfigure}
\usepackage{wrapfig}

% Fonts 
\usepackage[T1]{fontenc}
 % Required for including letters with accents
\usepackage[utf8]{inputenc}

% Italic section
\usepackage[it]{titlesec} 

% A garbage package you don't need except to create examples.
\usepackage{lipsum} 
 
% Headers 
\usepackage{fancyhdr}
\pagestyle{fancy}
% Header titles 
\renewcommand{\headrulewidth}{0.1pt}
% Header data 
\newlength\FHoffset
\setlength\FHoffset{0cm}
\addtolength\headwidth{0\FHoffset}
\fancyheadoffset{\FHoffset}

\fancyhead[LO,LE]{\emph{\fontsize{10}{12}\selectfont\nouppercase Title}} 
\fancyhead[RO,RE]{\fontsize{10}{12}\selectfont\nouppercase } % Empty

% Adjust the margins 
\usepackage[margin=0.8 in]{geometry}

% Default line 
%\renewcommand*\familydefault{\sfdefault} 

% Math equations 
\usepackage[sc]{mathpazo}

% Palatino needs more leading (space between lines)
\linespread{1.05}         

% Image oath 
\graphicspath{{images/}}

% Background image 
\usepackage{eso-pic}

% Required for including images
\usepackage{graphicx} 

% Required for manipulating the whitespace between and within lists
\usepackage{enumitem} 

% Used for inserting dummy 'Lorem ipsum' text into the template
\usepackage{lipsum} 

% For including math equations, theorems, symbols, etc
\usepackage{amsmath,amssymb,amsthm} 

% More descriptive referencing
\usepackage{varioref} 

% Add the colors 
\input{rgb.tex}

%%%%%%%%%%%%%%%%%%%%%%%%%%%%%%%%%%%%%%%%%
%	HYPERLINKS
%%%%%%%%%%%%%%%%%%%%%%%%%%%%%%%%%%%%%%%%%
\hypersetup{
	colorlinks = true, 
	breaklinks = true, 
	bookmarks = true,
	bookmarksnumbered,
	urlcolor = maroon, 
	linkcolor = blue-sky, 
	citecolor = blue-sky, 
	linktoc=page, 
	pdftitle={}, 
	pdfauthor={\textcopyright Author}, 
	pdfsubject={}, 
	pdfkeywords={}, 
	pdfcreator={pdfLaTeX}, % PDF Creator
pdfproducer={This PDF was generated based on Marwan Abdellah's Template} 
}

%%%%%%%%%%%%%%%%%%%%%%%%%%%%%%%%%%%%%%%%%
%	THEOREM STYLES
%%%%%%%%%%%%%%%%%%%%%%%%%%%%%%%%%%%%%%%%%
% Define theorem styles here based on the definition style (used for definitions and examples)
\theoremstyle{definition} 
\newtheorem{definition}{Definition}

% Define theorem styles here based on the plain style (used for theorems, lemmas, propositions)
\theoremstyle{plain} 
\newtheorem{theorem}{Theorem}

% Define theorem styles here based on the remark style (used for remarks and notes)
\theoremstyle{remark} 

% Add the capability to add a background image to the text.
%%%%%%%%%%%%%%%%%%%%%%%%%%%%%%%%%%%%%%%%%
% Latex Nice Article
% Add a background image to the text
%
% Author(s):
% Marwan Abdellah (abdelah.marwan@gmail.com)
%
% License:
% MIT License 
%
%%%%%%%%%%%%%%%%%%%%%%%%%%%%%%%%%%%%%%%%%
\newcommand
\BackgroundPic
{
	\put(0,0)
	{
		\parbox[b][\paperheight]{\paperwidth}
		{
			\vfill
			\centering
			\includegraphics[width=\paperwidth, height=\paperheight, keepaspectratio]
			{background-0}
			\vfill
		}
	}
}


% Let's begin the document. 
\begin{document}

% Specifiy the addition of the background to the text. 
%%%%%%%%%%%%%%%%%%%%%%%%%%%%%%%%%%%%%%%%%
% Latex Nice Article
% Adding a background image to the text
%
% Author(s):
% Marwan Abdellah (abdelah.marwan@gmail.com)
%
% License:
% MIT License 
%
%%%%%%%%%%%%%%%%%%%%%%%%%%%%%%%%%%%%%%%%%

% Add the background image to the first page only.  
\AddToShipoutPicture*{\BackgroundPic}

% If you wish to use the picture on multiple pages, skip the * and use this command 
% \AddToShipoutPicture{\BackgroundPic}

% Use this command to stop using the background picture
% \ClearShipoutPicture



% Add the cover Page.
%%%%%%%%%%%%%%%%%%%%%%%%%%%%%%%%%%%%%%%%%
% Latex Nice Article
%
% Author(s):
% Marwan Abdellah (abdelah.marwan@gmail.com)
%
% License:
% MIT License 
%
%%%%%%%%%%%%%%%%%%%%%%%%%%%%%%%%%%%%%%%%%

\begin{titlepage}
%------------------------------------------------------------------------ 
\vspace{5cm}
\centering{\large Small Title} \\ [0.1 cm]
\centering{\small Draft V.1} \\ [0.5 cm]
%------------------------------------------------------------------------

%------------------------------------------------------------------------
\begin{minipage}{\textwidth}
\centering{\Huge \textsc{{Big Title}}}
\end{minipage}
%------------------------------------------------------------------------

%------------------------------------------------------------------------
\vspace{1.5 cm}
\centering{\large {Author}} \\
\centering{{Affiliation}} \\
\centering \small {\textit{email@email.com}}
%------------------------------------------------------------------------

%------------------------------------------------------------------------
% Logos
\begin{figure}[h!]
\centering 
\includegraphics[scale=1]{sample-logo}  \\ [5 pt]
\end{figure}

\vspace{3cm}
\centering {\small September 2013} 

%------------------------------------------------------------------------
\end{titlepage}
%------------------------------------------------------------------------
%------------------------------------------------------------------------
%------------------------------------------------------------------------

%%%%%%%%%%%%%%%%%%%%%%%%%%%%%%%%%%%%%%%%%
% Lists 
%%%%%%%%%%%%%%%%%%%%%%%%%%%%%%%%%%%%%%%%%
\thispagestyle{empty}

% Add the table of contents, or comment it. 
\tableofcontents

% Add the list of figures, or comment it. 
\listoffigures
 
% Add the list of tables, or comment it. 
\listoftables

% Define the list of acronyms and add them. 
%%%%%%%%%%%%%%%%%%%%%%%%%%%%%%%%%%%%%%%%%
% Latex Nice Article
% List of acronyms. 
%
% Author(s):
% Marwan Abdellah (abdelah.marwan@gmail.com)
%
% License:
% MIT License 
%
%%%%%%%%%%%%%%%%%%%%%%%%%%%%%%%%%%%%%%%%%

\section*{List of Acronyms}
% Acronyms go here. 
\begin{acronym}
\acro{ABC}{Acronym Begins Capital}
\end{acronym}


\clearpage
%%%%%%%%%%%%%%%%%%%%%%%%%%%%%%%%%%%%%%%%%
 
% Adding the abstract page 
%%%%%%%%%%%%%%%%%%%%%%%%%%%%%%%%%%%%%%%%%
% Latex Nice Article
% List of acronyms
%
% Author(s):
% Marwan Abdellah (abdelah.marwan@gmail.com)
%
% License:
% MIT License 
%
%%%%%%%%%%%%%%%%%%%%%%%%%%%%%%%%%%%%%%%%%

\begin{abstract}
% Abstract goes here.
\lipsum[1]
\end{abstract}

% Keywords 
\paragraph*{Keywords}{\textit{keyword1, keyword2, keyword3,...}}

% Add a new page after the abstract. 
\newpage 

% Restore normal font if any interruptions. 
\normalfont 

%%%%%%%%%%%%%%%%%%%%%%%%%%%%%%%%%%%%%%%%%
% Article starts here !
%%%%%%%%%%%%%%%%%%%%%%%%%%%%%%%%%%%%%%%%%
% Article starts here !
%%%%%%%%%%%%%%%%%%%%%%%%%%%%%%%%%%%%%%%%%

%%%%%%%%%%%%%%%%%%%%%%%%%%%%%%%%%%%%%%%%%
% Introduction 
\section{Introduction} \label{section:introduction}
\lipsum[2-3]

A statement\footnote{Example of a footnote} requiring citation \cite{Figueredo:2009dg}. 
Some mathematics in the text: $\cos\pi=-1$ and $\alpha$.

\subsection{It is just a template !} \label{subsection:intro-subsection}
\lipsum[20]

%%%%%%%%%%%%%%%%%%%%%%%%%%%%%%%%%%%%%%%%%
%	Methods
\section{Methods} \label{section:methods}
Reference to Figure~\vref{fig:gallery}. % The \vref command specifies the location of the reference
The acronym \ac{ABC} is used here. 

\begin{figure}[tb]
\centering 
\includegraphics[width=0.5\columnwidth]{sample-image-2} 
\caption[An example of a floating figure]{An example of a floating figure (a reproduction from the \emph{Gallery of prints}, M.~Escher,\index{Escher, M.~C.} from \url{http://www.mcescher.com/}).} 
\label{fig:gallery} 
\end{figure}

\lipsum[5]

\subsection{Paragraphs}
\lipsum[6] 

\paragraph{Paragraph Description} 
\lipsum[7] 

\paragraph{Different Paragraph Description} 
\lipsum[8] 

\subsection{Math}
\lipsum[4] 

\begin{equation}
\cos^3 \theta =\frac{1}{4}\cos\theta+\frac{3}{4}\cos 3\theta
\label{eq:refname2}
\end{equation}

\lipsum[5] 

\begin{definition}[Gauss] 
To a mathematician it is obvious that
$\int_{-\infty}^{+\infty}
e^{-x^2}\,dx=\sqrt{\pi}$. 
\end{definition} 

\begin{theorem}[Pythagoras]
The square of the hypotenuse (the side opposite the right angle) is equal to the sum of the squares of the other two sides.
\end{theorem}

\begin{proof} 
We have that $\log(1)^2 = 2\log(1)$.
But we also have that $\log(-1)^2=\log(1)=0$.
Then $2\log(-1)=0$, from which the proof.
\end{proof}

%%%%%%%%%%%%%%%%%%%%%%%%%%%%%%%%%%%%%%%%%
% Results and Discussion 
\section{Results and Discussion} \label{section:results-and-discussion}
\lipsum[10]

\subsection{Results} \label{subsection:results}

\lipsum[11]

\subsubsection{Some Results} \label{subsubsection:some-results}
\lipsum[12] 

\begin{description}
\item[Word] Definition
\item[Concept] Explanation
\item[Idea] Text
\end{description}

\lipsum[12] 

\subsubsection{Table} \label{subsubsection:table}
\lipsum[13] 

\begin{table}[hbt]
\caption{Table of Grades}
\centering
\begin{tabular}{llr}
\toprule
\multicolumn{2}{c}{Name} \\
\cmidrule(r){1-2}
First name & Last Name & Grade \\
\midrule
John & Doe & $7.5$ \\
Richard & Miles & $2$ \\
\bottomrule
\end{tabular}
\label{tab:label}
\end{table}

Reference to Table~\vref{tab:label}. % The \vref command specifies the location of the reference

\subsection{Figure Composed of Subfigures} \label{subsection:compound}
Reference the figure composed of multiple subfigures as Figure~\vref{fig:esempio}. Reference one of the subfigures as Figure~\vref{fig:ipsum}. 

\lipsum[15-18]

\begin{figure}[tb]
\centering
\subfloat[A city market.]{\includegraphics[width=.45\columnwidth]{sample-image-1}} \quad
\subfloat[Forest landscape.]{\includegraphics[width=.45\columnwidth]{sample-image-3}\label{fig:ipsum}} \\
\subfloat[Mountain landscape.]{\includegraphics[width=.45\columnwidth]{sample-image-4}} \quad
\subfloat[A tile decoration.]{\includegraphics[width=.45\columnwidth]{sample-image-5}}
\caption[A number of pictures.]{A number of pictures with no common theme.}
\label{fig:esempio}
\end{figure}

%%%%%%%%%%%%%%%%%%%%%%%%%%%%%%%%%%%%%%%%%
% References 

% Plain style 
\bibliographystyle{unsrt}				

% Set the size to footnote size for small articles.
%\footnotesize

% Set the references in two columns. 
%\twocolumn

% The input .bib file
\bibliography{references}	 
%%%%%%%%%%%%%%%%%%%%%%%%%%%%%%%%%%%%%%%%%
\end{document}
%%%%%%%%%%%%%%%%%%%%%%%%%%%%%%%%%%%%%%%%%
% Article ends here !
%%%%%%%%%%%%%%%%%%%%%%%%%%%%%%%%%%%%%%%%%
% Article ends here !
%%%%%%%%%%%%%%%%%%%%%%%%%%%%%%%%%%%%%%%%%